
\chapter{\IfLanguageName{dutch}{Maatregelingen voor nummerplaatdetectie met Raspberry PI}{Implementation guide for ANPR with Raspberry PI}}
\label{ch:maatregelingenraspberrypi}

In deze sectie beoordelen we welke maatregelingen genomen moeten worden bij het implementeren van een ANPR systeem met oog op de parking aan de UGent.

Nummerplaatdetectie is al sterk geevolueerd sinds vroeger, maar heeft nog steeds enkele drawbacks. Zo spelen factoren zoals weer, belichting en plaatsing van de camera's een invloed op de nauwkeurigheid van de uitlezingen.

zoals camerahoek, resolutie, weerstomstandigheden, belichting, afbeeldingcompressie, tijd voor een uitlezen. Door het volgen van deze maatregelingen kan een werknemer nummerplaatdetectie installeren op een zo'n correct mogelijke manier.

\section{Maatregelingen}

\subsection{Detectie}
Hoe weet men wanneer een nummerplaat te detecteren?
Ultrasone sensors, autodetectors ondergronds

\subsection{Camera}
infrarode camera nacht
Detectie snachts \autocite{boonsin2017car}
De belangrijkste factor op nauwkeurigheid van ALPR is plaatsing en kwaliteit van de camera \autocite{openalprcameraplacement}.

\subsubsection{Belichting}
Infrarooed

\subsubsection{Camerahoek}
Uit een prototype van \textcite{arrieta2019prototype} bleek dat nummerplaten niet correct geidentificeerd werden bij een inclinatiehoek vanaf 30 graden. Het is dus eerder aanbevolen om een camera parallel te richten ipv. perpendiculair.

OpenALPR heeft opties om de hoek te calibreren \autocite{openalprdocumentation}

\subsubsection{Shuttersnelheid}
\textcite{guo2017vehicle} stelt een trainingsmodel voor dat rekening houdt met blur.

camera shutterspeed is de snelheid dat een camera foto's neemt. In een klaarlichte dag kan de shutterspeed zo'n 1/10000 seconden halen terwijl in het donker dit wel een volle seconde kan duren om genoeg licht te behalen.

Bij een lange shutterspeed kan het dus zijn dat een voertuig een meter vooruit is gereden, terwijl bij een kleine shutterspeed dit bv. maar een centimeter is. Een korte shutterspeed is dus interessant voor het implementeren van nummerplaatdetectie aangezien de auto minder ver is gereden en dus minder motion blur op de foto staat.

Het nadeel van een kleine shutterspeed te nemen is dat er veel minder licht aanwezig is op de foto's, wat de detectie dan weer omlaag brengt. Dit kan geremedieerd worden door LEDs te plaatsen die de nummerplaten oplichten.

\subsubsection{Belichting}
'S nachts is de belichting van de nummerplaten een stoorzender, de camera kan onmogelijk een kleine shutterspeed aanhouden en een genoeg belichte afbeelding krijgen. Hiervoor moet er dus een eigen belichting bijgezet worden.

Zelfs al wordt er belichting bijgezet zal de nummerplaat spijtig genoeg niet leesbaar zijn, dit komt doordat de koplampen van een auto ervoor zorgen dat de camera niet eens een nummerplaat meer ziet. Een algemene oplossing voor deze problemen is het gebruik van een IR-camera. Een IR-camera detecteert enkel IR-licht en heeft dus geen invloed van de koplampen van wagens. Verder is het voordeel hiervan dat IR-licht niet zichtbaar is voor het menselijk oog, en dus ongestoort snachts en overdag gebruikt kan worden.

**Foto**

\subsubsection{Pixels on target}

\subsection{Legale maatregelingen}
\subsubsection{Buitenlandse nummerplaten}
OpenALPR ondersteund herkenning van buitenlandse nummerplaten, maar kan 

\subsubsection{Motoren}


\subsection{OpenALPR}

\subsubsection{Configuratie}
\textcite{arrieta2019prototype} en \textcite{buhus2016automatic} concluderen beiden dat openalpr standaard goede resultaten biedt, maar nog hogere resulaten bereikt kunnen worden indien er verduidelijkt wordt welk type nummerplaten er verwacht wordt. Dit houdt factoren in zoals de juiste dataset van het land gebruiken en de volgorde van de kentekenkarakters aanduiden.

Door pattern matching toe te passen kunnen resultaten nog nauwkeuriger zijn. Hierbij wordt een reguliere expressie op alle top N resultaten uitgevoerd en worden de non-matching resultaten verworpen.

\subsubsection{Commerciele upgrades}
OpenALPR biedt commerciele versies van OpenALPR aan, deze zouden een hogere nauwkeurigheid bieden.

\section{Maatregelingen inzake UGent}
In dit deel gaan we na op welke wijze de maatregelingen toegepast kunnen worden op de campus van UGent.  
Motoren?
camera's aan ingangen?
waar camera plaatsen?
