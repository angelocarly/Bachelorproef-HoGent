
\chapter{\IfLanguageName{dutch}{Maatregelen voor ANPR}{Implementation guide for ANPR}}
\label{ch:maatregelenanpr}

In deze sectie beoordelen we welke maatregelen genomen moeten worden bij het implementeren van een ANPR systeem met oog op de parking aan de UGent.

Nummerplaatdetectie is al sterk geevolueerd sinds vroeger, maar heeft nog steeds enkele drawbacks. Zo spelen factoren zoals weer, belichting en plaatsing van de camera's een invloed op de nauwkeurigheid van de uitlezingen.

%TODO deze alinea klopt niet, begint met een halve zin
zoals camerahoek, resolutie, weerstomstandigheden, belichting, afbeeldingcompressie, tijd voor een uitlezen. Door het volgen van deze maatregelen kan een werknemer nummerplaatdetectie installeren op een zo'n correct mogelijke manier.

\section{Camera Plaatsing}

\subsection{Locatie van de camera}
Uit een prototype van \textcite{arrieta2019prototype} bleek dat nummerplaten niet correct geïdentificeerd werden bij een inclinatiehoek vanaf 30 graden. Het is dus aanbevolen om de camerahoek te beperken tot een kleine hoek.

Verder is het aangeraden om de camera hoger te plaatsen dan de koplampen van de auto, dit om te voorkomen dat de camera verblind wordt door het sterke licht.

\subsection{Camera oriëntatie}
De gedetecteerde nummerplaten horen parallel te staan met de randen van de afbeelding. Dit omdat de datasets voor OpenANPR getraint zijn met afbeeldingen van horizontale nummerplaten, maar niet van gedraaide. Indien het niet mogelijk is om een rechte afbeelding te nemen kan de afbeelding ook later gedraaid worden.

\subsection{Pixeldichtheid}
Het aantal pixels van de foto waaruit een nummerplaat bestaat is van belang voor OpenALPR voor een duidelijke herkenning. Indien een foto van veraf is genomen zal deze laag zijn en van dichtbij zal deze dan weer hoog zijn. OpenALPR verwacht voor Europeaanse nummerplaten minstens een wijdte van 75 pixels en een grootte meer dan 250 pixels verhoogt niet opmerkelijk de accuraatheid. \autocite{openalprcameraplacement}

\section{Camera instellingen}
De belangrijkste factor van een performant ANPR-systeem is een correct ingestelde camera. Het nemen van foto's is de eerste stap in het proces en indien hierop nummerplaten niet duidelijk zijn kan OpenALPR onmogelijk iets detecteren. In dit onderdeel worden de belangrijkste instellingen verduidelijkt die bijdragen tot een correcte foto voor gebruik bij nummerplaatdetectie.

\subsection{Shuttersnelheid}
%\textcite{guo2017vehicle} stelt een trainingsmodel voor dat rekening houdt met blur.

Camera shutterspeed is de snelheid dat een camera foto's neemt. In een klaarlichte dag kan de shutterspeed zo'n 1/10000 seconden halen terwijl in het donker dit wel een volle seconde kan duren om genoeg licht te behalen. \autocite{openalprcameraplacement}

Bij een lange shutterspeed kan het dus zijn dat een voertuig een meter vooruit is gereden, terwijl bij een kleine shutterspeed dit bv. maar een centimeter is. Een korte shutterspeed is dus interessant voor het implementeren van nummerplaatdetectie aangezien de auto minder ver is gereden en dus minder motion blur op de foto staat.

\begin{figure}[h!]
	\centering
	\begin{subfigure}[b]{0.4\linewidth}
		\includegraphics[width=\linewidth]{img/shutter-slow.jpg}
		\caption{Shutter speed van 1/60}
	\end{subfigure}
	\begin{subfigure}[b]{0.4\linewidth}
		\includegraphics[width=\linewidth]{img/shutter-fast.jpg}
		\caption{Shutter speed van 1/250}
	\end{subfigure}
	\label{fig:ntlpc}
	\caption{Vergelijking van verschillende shutterinstellingen. \autocite{easy2019shutter}}
\end{figure}

Het nadeel van een kleine shutterspeed te nemen is dat er veel minder licht aanwezig is op de foto's, wat de detectie dan weer omlaag brengt. Zo krijg je 's nachts bijna volledig zwarte foto's. Dit kan geremedieerd worden door belichting bij te plaatsen.

\subsection{Belichting}
%TODO in dit onderdeel zou er nog een stuk ontbreken
's Nachts is de belichting van de nummerplaten een stoorzender, de camera kan onmogelijk een kleine shutterspeed aanhouden en een genoeg belichte afbeelding krijgen. Hiervoor moet er dus een eigen belichting bijgezet worden.

Zelfs al wordt er belichting bijgezet zal de nummerplaat spijtig genoeg niet leesbaar zijn, dit komt doordat de koplampen van een auto ervoor zorgen dat de camera niet eens een nummerplaat meer ziet. Een algemene oplossing voor deze problemen is het gebruik van een IR-camera. Een IR-camera detecteerd enkel IR-licht en heeft dus geen invloed van de koplampen van wagens. Verder is het voordeel hiervan dat IR-licht niet zichtbaar is voor het menselijk oog, en dus ongestoort snachts en overdag gebruikt kan worden.

\begin{figure}[h!]
	\centering
	\begin{subfigure}[b]{0.4\linewidth}
		\includegraphics[width=\linewidth]{img/night-time-lpc-bad.png}
		\caption{Slecht geconfigureerde camera.}
	\end{subfigure}
	\begin{subfigure}[b]{0.4\linewidth}
		\includegraphics[width=\linewidth]{img/night-time-lpc-good.png}
		\caption{Correct geconfigureerde camera.}
	\end{subfigure}
	\caption{Vergelijking tussen camerainstellingen in de nacht. \autocite{axis2019license}}
\end{figure}
Infrarood, infraroodlamp, dag, nacht

\subsection{Depth of field}
Om scherpe afbeeldingen te verkrijgen moet de depth of field (DOF) van een camera correct ingesteld staan. Deze bepaald in welke range een afbeelding scherp is. Hoe groter de DOF, hoe verder de objecten in focus zijn. Bij afstanden onder de 10m is de DOF aan de kleine kant en moet deze heel nauwkeurig ingesteld worden. \autocite{axis2019license}

\begin{figure}[h!]
	\centering
	\includegraphics[width=\linewidth]{img/depth-of-field.png}
	\caption{Verduidelijking van depth of field. \autocite{axis2019license}}
\end{figure}


\section{Configuratie}

\subsection{Pattern matching}
\textcite{arrieta2019prototype} en \textcite{buhus2016automatic} concluderen beiden dat openalpr standaard goede resultaten biedt, maar nog hogere resulaten bereikt kunnen worden indien er verduidelijkt wordt welk type nummerplaten er verwacht wordt. Dit houdt factoren in zoals de juiste dataset van het land gebruiken en de volgorde van de kentekenkarakters aanduiden.

Door pattern matching toe te passen kunnen resultaten nog nauwkeuriger zijn. Hierbij wordt een reguliere expressie op alle top N resultaten uitgevoerd en worden de non-matching resultaten verworpen.

Een voorbeeld hiervan is op nummerplaten in Tsjechië, verkregen van \textcite{openalpr2015pattern}
Er wordt nummerplaatdetectie uitgevoerd op afbeelding \ref{patternmatching} met volgende regexpatronen die voorkomen in Tsjechië:
\begin{itemize}
	\item cz \#@\#\#\#\#\#
	\item cz \#@@\#\#\#\#
\end{itemize}

\begin{lstlisting}
[mhill@mhill-linux tmp]$ alpr -c eu -p cz cz_4s50233.jpg -n 40
Config file location provided via default location
plate0: 40 results
- 4S5O233     confidence: 90.947      pattern_match: 0
- 4S5O23      confidence: 87.8683     pattern_match: 0
- 4S5O23      confidence: 85.1644     pattern_match: 0
- 4S5O23S     confidence: 84.5445     pattern_match: 0
- 4S5O23B     confidence: 83.7395     pattern_match: 0
- 4S5O2S3     confidence: 83.3698     pattern_match: 0
- 4S5O23G     confidence: 83.1375     pattern_match: 0
- 4S50233     confidence: 83.0457     pattern_match: 1
- 4S5O2B3     confidence: 82.5635     pattern_match: 0
- 4S5O2       confidence: 82.0857     pattern_match: 0
- 4S5O2G3     confidence: 81.5684     pattern_match: 0
- 4S5O2J3     confidence: 81.0409     pattern_match: 0
- 4S5O2S      confidence: 80.2911     pattern_match: 0
... more results that do not match ...
\end{lstlisting}

\begin{figure}[h!]
	\centering
	\includegraphics[width=\linewidth]{img/pattern-matching.jpg}
	\caption{Pattern matching van OpenALPR \autocite{openalpr2015pattern}}
	\label{patternmatching}
\end{figure}

Hieruit is te zien dat bij de top 7 resultaten het middelste karakter als een O zien i.p.v. een 0. Door te kijken of de pattern matching succesvol was, zien we dat het achtste resultaat correct is.

\subsection{Buitenlandse nummerplaten}
OpenALPR heeft verscheidene configuraties voor Europa, Amerika en andere continenten. Door één van deze te selecteren wordt een ander model gekozen dat specifiek voor deze regio getraint is. Bij een keuze van de Europese databank worden er dan ook geen tot weinig fouten verwacht indien een buitenlandse nummerplaat gedetecteerd is.

\subsection{Commerciële upgrades}
OpenALPR wilt ook wel winst maken en biedt dan ook een commerciële versie van OpenALPR aan, deze zou een hogere nauwkeurigheid bieden in enkele cases waar de Open Source versie slechte presteert. \autocite{openalpr2019benchmark}
