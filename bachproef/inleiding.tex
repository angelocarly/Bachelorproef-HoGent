%%=============================================================================
%% Inleiding
%%=============================================================================

\chapter{\IfLanguageName{dutch}{Inleiding}{Introduction}}
\label{ch:inleiding}

%De inleiding moet de lezer net genoeg informatie verschaffen om het onderwerp te begrijpen en in te zien waarom de onderzoeksvraag de moeite waard is om te onderzoeken. In de inleiding ga je literatuurverwijzingen beperken, zodat de tekst vlot leesbaar blijft. Je kan de inleiding verder onderverdelen in secties als dit de tekst verduidelijkt. Zaken die aan bod kunnen komen in de inleiding~\autocite{Pollefliet2011}:

%\begin{itemize}
  %\item context, achtergrond
  %\item afbakenen van het onderwerp
  %\item verantwoording van het onderwerp, methodologie
  %\item probleemstelling
  %\item onderzoeksdoelstelling
  %\item onderzoeksvraag
  %\item \ldots
%\end{itemize}

UGent heeft probleem met de doorgangen van hun parking. Het huidige systeem met tokens is een oude technologie en betere varianten zouden moeten bestaan.

\section{\IfLanguageName{dutch}{Probleemstelling}{Problem Statement}}
\label{sec:probleemstelling}

%Uit je probleemstelling moet duidelijk zijn dat je onderzoek een meerwaarde heeft voor een concrete doelgroep. De doelgroep moet goed gedefinieerd en afgelijnd zijn. Doelgroepen als ``bedrijven,'' ``KMO's,'' systeembeheerders, enz.~zijn nog te vaag. Als je een lijstje kan maken van de personen/organisaties die een meerwaarde zullen vinden in deze bachelorproef (dit is eigenlijk je steekproefkader), dan is dat een indicatie dat de doelgroep goed gedefinieerd is. Dit kan een enkel bedrijf zijn of zelfs één persoon (je co-promotor/opdrachtgever).
Vandaag de dag kampt UGent met problemen omtrent hun toegangssysteem aan de campus Coupure en campus Sterre. Het huidige systeem dat gebruikt maakt van tokens is niet efficient en zorgt voor veel extra werk. Daarom zouden Vado Solutions en UGent overwegen om over te stappen naar een nieuw systeem dat gebruikt maakt van nummerplaatdetectie. Dit met een Raspberry PI omdat deze hardware goedkoop is.

In dit onderzoek zal nagegaan worden of zulk systeem wel degelijk deftige resultaten kan bieden op deze sites, met een Raspberry PI.

\section{\IfLanguageName{dutch}{Onderzoeksvraag}{Research question}}
\label{sec:onderzoeksvraag}

%Wees zo concreet mogelijk bij het formuleren van je onderzoeksvraag. Een onderzoeksvraag is trouwens iets waar nog niemand op dit moment een antwoord heeft (voor zover je kan nagaan). Het opzoeken van bestaande informatie (bv. ``welke tools bestaan er voor deze toepassing?'') is dus geen onderzoeksvraag. Je kan de onderzoeksvraag verder specifiëren in deelvragen. Bv.~als je onderzoek gaat over performantiemetingen, dan 
Bij dit onderzoek bekomen we drie onderzoeksvragen:
\begin{itemize}
	\item Is nummerplaatdetectie een haalbare techniek omtrent privacy en GDPR?
	\item Welke maatregelingen moeten er genomen worden om succesvol nummerplaatdetectie te implementeren?
	\item Kan men nummerplaatdetectie succesvol uitvoeren met een Raspberry PI op de campus Coupure en Sterre van UGent?
\end{itemize}

\section{\IfLanguageName{dutch}{Onderzoeksdoelstelling}{Research objective}}
\label{sec:onderzoeksdoelstelling}

%Wat is het beoogde resultaat van je bachelorproef? Wat zijn de criteria voor succes? Beschrijf die zo concreet mogelijk. Gaat het bv. om een proof-of-concept, een prototype, een verslag met aanbevelingen, een vergelijkende studie, enz.

Dit onderzoek wordt als succesvol beschouwd indien een correcte manier van werk is gevonden om aan de privacywetgevingen te voldoen, een duidelijk overzicht van maatregelingen voor ANPR is bekomen en wanneer een performantie is gevonden voor nummerplaatdetectie aan de campus Coupure en Sterre van UGent.

\section{\IfLanguageName{dutch}{Opzet van deze bachelorproef}{Structure of this bachelor thesis}}
\label{sec:opzet-bachelorproef}

% Het is gebruikelijk aan het einde van de inleiding een overzicht te
% geven van de opbouw van de rest van de tekst. Deze sectie bevat al een aanzet
% die je kan aanvullen/aanpassen in functie van je eigen tekst.

De rest van deze bachelorproef is als volgt opgebouwd:

In Hoofdstuk~\ref{ch:stand-van-zaken} wordt een overzicht gegeven van de stand van zaken binnen het onderzoeksdomein, op basis van een literatuurstudie.

In Hoofdstuk~\ref{ch:methodologie} wordt de methodologie toegelicht en worden de gebruikte onderzoekstechnieken besproken om een antwoord te kunnen formuleren op de onderzoeksvragen.

In Hoofdstuk~\ref{ch:wetgeving-nummerplaatdetectie} wordt er nagegaan waarop er gelet moet worden bij het implementeren van nummerplaatdetectie als toegangssysteem op vlak van privacywetgevingen.

In Hoofdstuk~\ref{ch:maatregelingenanpr} wordt er nagegaan welke maatregelingen er genomen moeten worden om een zo performant mogelijke implementatie van nummerplaatdetectie te maken.

In Hoofdstuk~\ref{ch:praktischeUitvoering} wordt een onderzoek uitgevoerd aan de hand van de vooropgestelde maatregelingen aan de campus Sterre en Coupure van UGent. Hieruit zal blijken of nummerplaatdetectie met een Raspberry PI mogelijk is op deze locatie.

In Hoofdstuk~\ref{ch:conclusie}, tenslotte, wordt de conclusie gegeven en een antwoord geformuleerd op de onderzoeksvragen. Daarbij wordt ook een aanzet gegeven voor toekomstig onderzoek binnen dit domein.