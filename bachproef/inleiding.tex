%%=============================================================================
%% Inleiding
%%=============================================================================

\chapter{\IfLanguageName{dutch}{Inleiding}{Introduction}}
\label{ch:inleiding}

%De inleiding moet de lezer net genoeg informatie verschaffen om het onderwerp te begrijpen en in te zien waarom de onderzoeksvraag de moeite waard is om te onderzoeken. In de inleiding ga je literatuurverwijzingen beperken, zodat de tekst vlot leesbaar blijft. Je kan de inleiding verder onderverdelen in secties als dit de tekst verduidelijkt. Zaken die aan bod kunnen komen in de inleiding~\autocite{Pollefliet2011}:

%\begin{itemize}
  %\item context, achtergrond
  %\item afbakenen van het onderwerp
  %\item verantwoording van het onderwerp, methodologie
  %\item probleemstelling
  %\item onderzoeksdoelstelling
  %\item onderzoeksvraag
  %\item \ldots
%\end{itemize}

%Vandaag de dag heeft UGent enkele problemen met de toegangscontrole van hun parking. Het huidige systeem met tokens is een oude technologie en is aan vernieuwing toe.

\section{\IfLanguageName{dutch}{Probleemstelling}{Problem Statement}}
\label{sec:probleemstelling}

%Uit je probleemstelling moet duidelijk zijn dat je onderzoek een meerwaarde heeft voor een concrete doelgroep. De doelgroep moet goed gedefinieerd en afgelijnd zijn. Doelgroepen als ``bedrijven,'' ``KMO's,'' systeembeheerders, enz.~zijn nog te vaag. Als je een lijstje kan maken van de personen/organisaties die een meerwaarde zullen vinden in deze bachelorproef (dit is eigenlijk je steekproefkader), dan is dat een indicatie dat de doelgroep goed gedefinieerd is. Dit kan een enkel bedrijf zijn of zelfs één persoon (je co-promotor/opdrachtgever).

Vandaag de dag kampt UGent met problemen omtrent hun toegangssysteem van hun parking aan de campus Coupure en campus Sterre. Het huidige systeem dat gebruikt maakt van tokens is niet efficiënt en zorgt voor veel extra werk. Dit is werk zoals:
\begin{itemize}
	\item De tokens moeten telkens afgehaald worden aan het onthaal om de campussen te kunnen verlaten.
	\item De tokenslikkers moeten regelmatig geleegd worden indien de tokenslikkers vol zijn.
	\item De tokens zijn relatief duur om bij te maken.
	\item De tokens worden snel kwijtgeraakt.
\end{itemize}
Door deze problemen overweegt UGent om op deze locaties over te stappen naar een beter systeem dat milieubewust is, beperkt in kostprijs is, en een goed gebruiksgemak heeft. 

Hierop biedt VaDo Solutions een innovatief systeem aan dat gebruik maakt van nummerplaatdetectie. Deze zou gebruik maken van een Raspberry PI in combinatie met een open-source library genaamd OpenALPR. Welke al reeds bevestigd zijn dat ze goede resultaten kunnen opleveren \autocite{figuerola2016automated}. Maar of deze resultaten ook haalbaar zijn op UGent kan hiermee niet bevestigd worden. Dit zal dit onderzoek nagaan.

Vervolgens heerst er ook nog enkele onduidelijkheid over hoe de GDPR inspeelt op een dergelijk systeem en welke maatregelen hierrond getroffen moeten worden.

\section{\IfLanguageName{dutch}{Onderzoeksvraag}{Research question}}
\label{sec:onderzoeksvraag}

%Wees zo concreet mogelijk bij het formuleren van je onderzoeksvraag. Een onderzoeksvraag is trouwens iets waar nog niemand op dit moment een antwoord heeft (voor zover je kan nagaan). Het opzoeken van bestaande informatie (bv. ``welke tools bestaan er voor deze toepassing?'') is dus geen onderzoeksvraag. Je kan de onderzoeksvraag verder specifiëren in deelvragen. Bv.~als je onderzoek gaat over performantiemetingen, dan 
Dit onderzoek zal nagaan in hoeverre het mogelijk is nummerplaatdetectie op te stellen op de Campus Coupure en Campus Sterre van UGent. Hiervoor wordt er bekeken in welke mate de GDPR invloed heeft op nummerplaatdetectie. Daarnaast zullen er maatregelen opgesteld worden om een nauwkeurige detectie te verkrijgen, waarop er vervolgens een testopstelling gemaakt wordt om na te gaan of een dit wel degelijk mogelijk is op UGent.

Zo bekomen we volgende drie onderzoeksvragen:
\begin{itemize}
	\item Is nummerplaatdetectie een haalbare techniek omtrent GDPR?
	\item Welke maatregelen moeten er genomen worden om succesvol nummerplaatdetectie te implementeren?
	\item Kan men nummerplaatdetectie succesvol uitvoeren met een Raspberry Pi op de campus Coupure en Sterre van UGent?
\end{itemize}

Deze vragen zullen doorheen dit onderzoek beantwoord worden.

\section{\IfLanguageName{dutch}{Onderzoeksdoelstelling}{Research objective}}
\label{sec:onderzoeksdoelstelling}

%Wat is het beoogde resultaat van je bachelorproef? Wat zijn de criteria voor succes? Beschrijf die zo concreet mogelijk. Gaat het bv. om een proof-of-concept, een prototype, een verslag met aanbevelingen, een vergelijkende studie, enz.

Dit onderzoek heeft tot doel om een correcte voorstelling te geven over hoe nauwkeurig nummerplaatdetectie met een Raspberry Pi en OpenALPR uitgevoerd kan worden op UGent. Dit zal gedaan worden door een prototype hiervan op te stellen en uit te testen aan de uitgangen van UGent Campus Sterre en Campus Coupure. Door de resultaten van dit onderzoek zal VaDo-Solutions kunnen beslissen of dit wel of niet de gewenste technologie is die zij willen gebruiken.

Vervolgens wordt er verwacht dat een duidelijk overzicht gegeven wordt van maatregelen die getroffen kunnen worden om een dergelijk systeem te implementeren. Deze maatregelen zullen zijn op vlak van hardware- en software-instellingen, zowel als fysieke variabelen zoals locatie en richting. Zo kan een installateur op een vlotte manier inzicht verkrijgen welke factoren een grote bijdrage leveren aan een nauwkeurig ANPR-systeem.

Ten laatste is het de bedoeling om een bondige uitleg te hebben op welke vlakken de GDPR invloed heeft op dit soort camerasysteem. Hiermee kan een ontwikkelaar weten aan welke voorwaarden zijn opstelling moet voldoen zodat hij geen risico loopt op het overtreden van deze wetgevingen.

\section{\IfLanguageName{dutch}{Opzet van deze bachelorproef}{Structure of this bachelor thesis}}
\label{sec:opzet-bachelorproef}

% Het is gebruikelijk aan het einde van de inleiding een overzicht te
% geven van de opbouw van de rest van de tekst. Deze sectie bevat al een aanzet
% die je kan aanvullen/aanpassen in functie van je eigen tekst.

De rest van deze bachelorproef is als volgt opgebouwd:

In Hoofdstuk~\ref{ch:stand-van-zaken} wordt een overzicht gegeven van de stand van zaken binnen het onderzoeksdomein, op basis van een literatuurstudie.

In Hoofdstuk~\ref{ch:methodologie} wordt de methodologie toegelicht en worden de gebruikte onderzoekstechnieken besproken om een antwoord te kunnen formuleren op de onderzoeksvragen.

In Hoofdstuk~\ref{ch:wetgeving-nummerplaatdetectie} wordt er nagegaan waarop er gelet moet worden bij het implementeren van nummerplaatdetectie als toegangssysteem op vlak van privacywetgevingen.

In Hoofdstuk~\ref{ch:maatregelenanpr} wordt er nagegaan welke maatregelen er genomen moeten worden om een zo performant mogelijke implementatie van nummerplaatdetectie te maken.

In Hoofdstuk~\ref{ch:praktischeUitvoering} wordt een steekproef uitgevoerd aan de hand van de vooropgestelde maatregelen aan de campus Sterre en Coupure van UGent. Hieruit zal blijken of nummerplaatdetectie met een Raspberry PI wel degelijk mogelijk is op deze locatie.

In Hoofdstuk~\ref{ch:conclusie}, tenslotte, wordt de conclusie gegeven en een antwoord geformuleerd op de onderzoeksvragen. Daarbij wordt ook een aanzet gegeven voor toekomstig onderzoek binnen dit domein.