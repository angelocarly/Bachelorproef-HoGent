%%=============================================================================
%% Methodologie
%%=============================================================================

\chapter{\IfLanguageName{dutch}{Methodologie}{Methodology}}
\label{ch:methodologie}

%%  Hoe ben je te werk gegaan? Verdeel je onderzoek in grote fasen, en
%% licht in elke fase toe welke stappen je gevolgd hebt. Verantwoord waarom je
%% op deze manier te werk gegaan bent. Je moet kunnen aantonen dat je de best
%% mogelijke manier toegepast hebt om een antwoord te vinden op de
%% onderzoeksvraag.

In dit onderzoek wensen we inzicht te verkrijgen of een ANPR-systeem succesvol geïmplementeerd kan worden aan de parking van UGent op de campussen Sterre en Coupure. Ook wordt onderzocht welke maatregelen getroffen dienen te worden om zo'n systeem aan de GDPR te laten voldoen. Deze doelen werden opgesplitst in drie fasen en worden beschreven in de volgende hoofdstukken.

\section{Richtlijnen omtrent GDPR bij nummerplaatdetectie}
Sinds de GDPR vorig jaar is ingevoerd, moeten bedrijven veel meer rekening houden met hoe ze data verwerken. Deze richtlijnen zijn allemaal te vinden in de wettekst van de GDPR zelf, maar om deze te verduidelijken worden deze opgesomd.

Op basis van de richtlijnen die in dit hoofdstuk omschreven worden kan een ontwikkelaar inzicht krijgen in hoe een nummerplaatdetectiesysteem ingevoerd kan worden. Indien zo'n systeem deze richtlijnen niet volgt zal deze ook niet voldoen aan de GDPR. 

\section{Maatregelen voor nummerplaatdetectie met Raspberry Pi}
Een ANPR-systeem opzetten op een Raspberry Pi is niet vanzelfsprekend aangezien ANPR-systemen normaal met dure hardware worden geïnstalleerd. Om toch nauwkeurige resultaten te kunnen boeken, zal er in dit hoofdstuk beschreven worden wat de optimale waarden zijn in kwestie van camera-instellingen, plaatsing van de camera, netwerk.

A.d.h.v. deze maatregelen kan een ontwikkelaar een ANPR-systeem configureren met een zo hoog mogelijke nauwkeurigheid.

\section{Praktische uitvoering van nummerplaatdetectie op UGent}
Vervolgens kan er a.d.h.v. de vooropgestelde maatregelen getest worden of ANPR met een Raspberry Pi haalbaar is op UGent. Hiervoor zullen er op de campussen van de UGent foto's genomen worden met de Pi-Cam van voertuigen die de parkings willen verlaten. Hierbij zal er gekeken worden welke uitvoeringstijd de detectie nodig heeft. Achteraf wordt er per foto gecontroleerd of de gefotografeerde nummerplaat wel degelijk juist is gedetecteerd. Indien deze nauwkeurigheid hoog genoeg is, kan er besloten worden dat ANPR met een Raspberry PI een haalbare techniek is op UGent.

Voor het maken van de foto's zal gebruik gemaakt worden van de PiNoIR Camera, dit is een camera voor de Raspberry Pi die geen filtering heeft op infrarood licht, wat het optimaal maakt voor gebruik in donkere situaties \autocite{raspberrypisitemodelpinoir}.