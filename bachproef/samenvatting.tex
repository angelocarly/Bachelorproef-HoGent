%%=============================================================================
%% Samenvatting
%%=============================================================================

% De "abstract" of samenvatting is een kernachtige (~ 1 blz. voor een
% thesis) synthese van het document.
%
% Deze aspecten moeten zeker aan bod komen:
% - Context: waarom is dit werk belangrijk?
% - Nood: waarom moest dit onderzocht worden?
% - Taak: wat heb je precies gedaan?
% - Object: wat staat in dit document geschreven?
% - Resultaat: wat was het resultaat?
% - Conclusie: wat is/zijn de belangrijkste conclusie(s)?
% - Perspectief: blijven er nog vragen open die in de toekomst nog kunnen
%    onderzocht worden? Wat is een mogelijk vervolg voor jouw onderzoek?
%
% LET OP! Een samenvatting is GEEN voorwoord!

%%---------- Nederlandse samenvatting -----------------------------------------

\IfLanguageName{english}{%
\selectlanguage{dutch}
\chapter*{Samenvatting}
%tekst
\selectlanguage{english}
}{}

%%---------- Samenvatting -----------------------------------------------------
% De samenvatting in de hoofdtaal van het document

\chapter*{\IfLanguageName{dutch}{Samenvatting}{Abstract}}

Tesamen met Vado-Solutions werd besloten om een onderzoek uit te voeren naar de mogelijkheid van nummerplaatdetectie op UGent op de Campussen Sterre en Coupure. De reden hiervoor was dat het huidige systeem van fysieke tokens die ingeworpen moeten worden niet gebruiksvriendelijk is en zou kunnen genieten van een vernieuwing. De nieuwe implementatie zou gebruiksvriendelijk, milieuvriendelijk en goedkoop moeten zijn.

Vado-Solutions had de interesse om hiervoor een Raspberry Pi met PiNoIR camera en het open-source framework van OpenALPR te gebruiken. Deze hardware is uiterst goedkoop en zou een degelijke oplossing kunnen vormen, maar een werkelijke implementatie loopt niet altijd zoals gepland en er is geen zekerheid of dit wel degelijke resultaten zou opleveren. Dit onderzoek gaat dit na.

Het eerste deel van het onderzoek was het nagaan of een dergelijk systeem wel degelijk als legaal kan aanschouwd worden onder de huidige wetgeving van de GDPR. Deze was met oog op de voorgestelde implementatie van Vado-solutions zeker haalbaar. Iedere vorm van verwerken van persoonsgegevens dient gerechtigd te zijn en mogen enkel voor duidelijk omschreven doeleinden gebruikt worden. Verder is het verplicht om deze gegevens te kunnen opleveren of aanpassen indien gewenst van de eigenaar van de gegevens of enige autoriteiten. Deze verplichtingen hebben een grote implementatie- en onderhoudskost, wat niet gewenst is in een goedkope oplossing. Gelukkig blijkt dat indien er geen persoonsgegevens bijgehouden worden, niet aan deze voorwaarden voldoen kunnen worden. Dit brengt kosten omlaag en maakt een dergelijke implementatie mogelijk.

In het tweede deel zijn de mogelijke valkuilen voor een nummerplaatdetectiesysteem onderzocht in een literatuurstudie. Hieruit bleek dat een degelijke cameraconfiguratie van uiterst belang is. Een te lage resolutie, overbelichting of slechte invalshoek kunnen een drastische invloed hebben op de bekomen resultaten, en dienen met zorg ingesteld te worden.

Ten laatste werd er onderzocht of een fysieke implementatie wel degelijk haalbaar was op de locaties van UGent. Om dit te onderzoeken werden uitrijdende voertuigen op de vier uitgangen gefotografeerd met een Raspberry Pi met een PiNoir camera. Deze werden vervolgens verwerkt met OpenALPR, waarna de resultaten geanalyseerd werden. Uit de resultaten bleek dat gemiddeld 94.7\% van de voertuigen correct geïdentificeerd konden worden in rustige omstandigheden in daglicht. Deze resultaten zijn een geslaagd resultaat voor een steekproef en stellen de weg naar verder onderzoek open.

Uit dit onderzoek is gebleken dat nummerplaatdetectie mbv. een Raspberry Pi met PiNoIR-cam en OpenALPR wel degelijk mogelijk is om toegepast te worden op de uitgangen van UGent Campus Coupure en Campus Sterre. De voorgestelde implementatie van Vado-Solutions werd bevestigd om weinig tot geen invloed te ondervinden van de GDPR en een steekproef met veelbelovende resultaten was bekomen.

De resultaten van dit onderzoek stellen de weg open naar een breder onderzoek. Er is nog nood aan inzicht of deze resultaten ook 's nachts kunnen bekomen worden, zowel als de opties om een automatische cameratriggering te bekomen.
