%%=============================================================================
%% Conclusie
%%=============================================================================

\chapter{Conclusie}
\label{ch:conclusie}

% Trek een duidelijke conclusie, in de vorm van een antwoord op de
% onderzoeksvra(a)g(en). Wat was jouw bijdrage aan het onderzoeksdomein en
% hoe biedt dit meerwaarde aan het vakgebied/doelgroep? 
% Reflecteer kritisch over het resultaat. In Engelse teksten wordt deze sectie
% ``Discussion'' genoemd. Had je deze uitkomst verwacht? Zijn er zaken die nog
% niet duidelijk zijn?
% Heeft het onderzoek geleid tot nieuwe vragen die uitnodigen tot verder 
%onderzoek?

Het doel van dit onderzoek was het nagaan of nummerplaatdetectie een haalbare technologie was aan de UGent Campus Sterre en Campus Coupure. Dit met behulp van een Raspberry Pi met PiCam en de open-source implementatie van OpenALPR die een goedkope oplossing zouden kunnen bieden. Om deze vraag te beantwoorden werd onderzocht of een dergelijke implementatie mogelijk is met de intrede van de GDPR, en of deze wel degelijk goede resultaten kan opleveren op de uitgangen van de campussen zelf.

\section{Is nummerplaatdetectie een haalbare techniek omtrent GDPR?}
Om de eerste vraag te beantwoorden werd een literatuurstudie uitgevoerd over de GDPR zelf, dit hield in hoe deze werkt en hoe deze een ANPR-systeem beïnvloedt. Uit de studie bleek dat de wetgeving een zeer grote invloed heeft op een ANPR-detectie implementatie. Foto's en andere persoonsgegevens zijn verplicht zo min mogelijk verwerkt te worden en dit enkel indien dit gerechtvaardigd is. Verder moet het mogelijk zijn voor een betrokkene om deze op te vragen, te corrigeren of te verwijderen.

Een wettelijke implementatie van een ANPR-systeem is wellicht haalbaar. De wettelijke gronden staan de verwerking van de foto's toe op de grond van het gerechtvaardigde belang zolang deze een duidelijke doelbinding hebben en niet voor andere zaken gebruikt worden. Functionaliteiten zoals het aanpassen van persoonsgegevens of het beveiligen van de gegevens zouden normaal gezien een hoge implementatiekost hebben om hier een systeem rond te bouwen, maar door persoonsgegevens tot een absoluut minimum te houden en de betrokkene niet identificeerbaar te maken is deze functionaliteit niet vereist en dekt men implementatiekosten. Dit maakt het mogelijk om een implementatie goedkoop te maken en aldus haalbaarder.

\section{Welke maatregelen moeten er genomen worden om succesvol nummerplaatdetectie te implementeren?}
Uit de tweede literatuurstudie naar maatregelen rond ANPR kwam naar voor dat nummerplaatdetectie vooral afhankelijk is van de gebruikte camera-instellingen en de OpenALPR configuratie. Zo speelt de locatie van de camera een groot belang tegen de interferentie van de koplampen van voertuigen. De studie in hoofdstuk \ref{ch:maatregelenanpr} kan dan ook gebruikt worden als richtlijnen voor een fysieke implementatie.

\section{Kan men nummerplaatdetectie succesvol uitvoeren met een Raspberry Pi op de campus Coupure en Sterre van UGent?}
Op basis van de verkregen richtlijnen werd een opstelling gemaakt. Deze haalde een verwachte nauwkeurigheid van gemiddeld 94.7\%, wat overeenkomt met het onderzoek van \textcite{figuerola2016automated}, waar men in optimale omstandigheden en gelijkaardige technologieën een nauwkeurigheid van 94.4\% behaalde. Deze resultaten doen vermoeden dat een ANPR-systeem wel degelijk mogelijk is aan de uitgangen van UGent. Deze resultaten stellen de weg open naar breder onderzoek op deze locaties.

Voor de opstelling zelf is een relatief goedkope oplossing gevonden. Op twee van de uitgangen was het mogelijk om de camera simpelweg op de metalen constructie van de hefboom te plaatsen, wat kosten omlaag brengt. Voor de uitgang aan de Campus Sterre De Pintelaan was dit helaas niet mogelijk door de vele inrijrichtingen en de interferentie van het zonlicht. Hierdoor is het essentieel om een verhoging of een paal te plaatsen zodat de camera verhoogd kan worden om een degelijk resultaat te verkrijgen. Dit zal de kosten aan deze uitgang doen stijgen i.v.m. andere uitgangen.

Ten laatste moet er benadrukt worden dat deze resultaten behaald zijn onder normale weersomstandigheden in zonlicht en lichte regen. Verder onderzoek is nodig om te bevestigen of een dergelijke opstelling succesvol is 's nachts of in hevige weersomstandigheden. Alsook om de detectie aan de uitgang op de Campus Coupure - Uitgang Coupure Links te bepalen.