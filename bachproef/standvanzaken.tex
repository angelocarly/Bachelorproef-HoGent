\chapter{Stand van zaken}
\label{ch:stand-van-zaken}

% Tip: Begin elk hoofdstuk met een paragraaf inleiding die beschrijft hoe
% dit hoofdstuk past binnen het geheel van de bachelorproef. Geef in het
% bijzonder aan wat de link is met het vorige en volgende hoofdstuk.

% Pas na deze inleidende paragraaf komt de eerste sectiehoofding.

%Dit hoofdstuk bevat je literatuurstudie. De inhoud gaat verder op de inleiding, maar zal het onderwerp van de bachelorproef *diepgaand* uitspitten. De bedoeling is dat de lezer na lezing van dit hoofdstuk helemaal op de hoogte is van de huidige stand van zaken (state-of-the-art) in het onderzoeksdomein. Iemand die niet vertrouwd is met het onderwerp, weet er nu voldoende om de rest van het verhaal te kunnen volgen, zonder dat die er nog andere informatie moet over opzoeken \autocite{Pollefliet2011}.

%Je verwijst bij elke bewering die je doet, vakterm die je introduceert, enz. naar je bronnen. In \LaTeX{} kan dat met het commando \texttt{$\backslash${textcite\{\}}} of \texttt{$\backslash${autocite\{\}}}. Als argument van het commando geef je de ``sleutel'' van een ``record'' in een bibliografische databank in het Bib\TeX{}-formaat (een tekstbestand). Als je expliciet naar de auteur verwijst in de zin, gebruik je \texttt{$\backslash${}textcite\{\}}.
%Soms wil je de auteur niet expliciet vernoemen, dan gebruik je \texttt{$\backslash${}autocite\{\}}. In de volgende paragraaf een voorbeeld van elk.

%\textcite{Knuth1998} schreef een van de standaardwerken over sorteer- en zoekalgoritmen. Experten zijn het erover eens dat cloud computing een interessante opportuniteit vormen, zowel voor gebruikers als voor dienstverleners op vlak van informatietechnologie~\autocite{Creeger2009}.

\section{Situatie parking UGent}
% Huidige situatie UGent
Het huidige toegangssysteem aan UGent is een systeem op basis van tokens. Een bezoeker rijdt de parking op zonder enige checks. Vervolgens bezoekt hij de campus en vraagt een token om de campus te verlaten. Ten laatste rijdt hij zijn wagen naar de slagboom en geeft zijn token in de gepaste tokenslikker.

\subsection{Huidige technologieen}

\subsubsection{Tokens}
\begin{itemize}
	\item verouderd simpelweg, moet buiten
	\item duur
	\item Personeelskosten: legen van tokenslikkers, tokens uitdelen
\end{itemize}

\subsubsection{RFID}
studentenkaarten zijn rfid (?), maar niet iedereen heeft studentenkaart terwijl wel ieder voertuig nummerplaat heeft.
extra kosten! scanners, niet simpel te integreren (?), gelimiteerde administratieve trucjes.

\subsection{Mogelijke nieuwe technologieen}

\subsubsection{ANPR}
	\textcite{aalsalem2015automated} beschrijft wat onveilig is aan rfid. (kopieren van kaarten)
	heel modulair. mensen kunnen een dagpas krijgen, toegang kan simpel gerevoked worden. Weinig kosten. Nauwkeurigheid (?)
	Wat met snachts, regen, brommers, non-standaard nummerplaten. Is dit nog steeds de moeite dan?

\subsubsection{RFID}

%TODO OP TE ZOEKEN:
rfid veiligheid
rfid voordelen
rfid kost
anpr voordelen?
anpr nadelen?
tokens nadelen?
tokens voordelen?
	
\section{Privacy en GDPR}
\label{sec:privacy-en-gdpr}

Sinds 25 Mei 2018 is de General Data Protection Regulation (GDPR) in gang gezet, een regulatie die ingevoerd is om het huidige  en toekomstige digitale tijdperk veiliger te maken voor alle EU inwoners. 
Deze wetgeving is gedreven op het concept dat privacy een mensenrecht is, en dat online-data ook zo behandeld moet worden. Dit houdt in data die direct of indirect gelinkt kan worden aan een individu zoals locatie-data, cookies en ip-adressen.\autocite{goddard2017eu}

\subsection{Gevolgen voor bedrijven}


%TODO gevolgen voor bedrijven

\subsection{ANPR camera systemen}
\label{anpr-cameras}
%TODO Hoe heeft gdpr invload op onze ANPR camera's
%TODO ANPR verduidelijken in eerdere tekst.

(?) Zolang er geen persoonlijke data bijgehouden is zijn we niet onderhevig aan van gdpr zoals toestemming van klant (?)
Automatic Number Plate Recognition (ANPR)

\subsection{GDPR toegepast als toegangssysteem aan UGent}
