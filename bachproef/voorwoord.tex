%%=============================================================================
%% Voorwoord
%%=============================================================================

\chapter*{\IfLanguageName{dutch}{Woord vooraf}{Preface}}
\label{ch:voorwoord}

%% Het voorwoord is het enige deel van de bachelorproef waar je vanuit je
%% eigen standpunt (``ik-vorm'') mag schrijven. Je kan hier bv. motiveren
%% waarom jij het onderwerp wil bespreken.
%% Vergeet ook niet te bedanken wie je geholpen/gesteund/... heeft

Deze bachelorproef "Analyse van nummerplaatdetectie aan de parkingen op UGent Campus Sterre en Campus Coupure"\ werd geschreven met als doel het voltooien van mijn opleiding Toegaste Informatica aan de Hogeschool Gent. Dit onderwerp was tot stand gekomen nadat ik kennis vernam dat Vado-Solutions een dergelijk systeem aan het overwegen was in samenspraak met UGent. Dit sprak mij direct aan als een uitdagend onderwerp waar ik mijn kennis over GDPR en Artificial Intelligence kan bijschaven. Ik ben dan ook zeer dankbaar dat Vado-Solutions mij deze kans geschonken heeft.

Graag zou ik ook even de tijd nemen om enkele mensen te bedanken zonder wiens hulp en begeleiding deze bachelorproef niet tot stand zou gekomen zijn.

Eerst en vooral wil ik mijn co-promotors, Gino Van Dorpe en Wannes Van Dorpe, bedanken voor de steun en feedback in het maken van deze bachelorproef, evenals deze mooie kans om dit onderwerp te mogen uitvoeren.

Ook wil ik mijn promotor, Lotte Van Steenberghe bedanken om mij met nuttige feedback en ondersteuning goed op weg te zetten doorheen dit verhaal.

Ten laatste zou ik ook mijn ouders en medestudenten willen bedanken. In het bijzonder Simon Anckaert, Pieter Vandendriessche en Shauni Van De Velde, om mij the helpen met morele steun en wijze feedback doorheen deze periode.