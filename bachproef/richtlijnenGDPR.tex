
\chapter{\IfLanguageName{dutch}{Wetgeving omtrent nummerplaatdetectie}{Technical details about ANPR in the GDPR}}
\label{ch:wetgeving-nummerplaatdetectie}

Algemene Verordening Gegevensbescherming (AVG)

%gemaakt om gegevens omtrent natuurlijke personen te beschermen en is van toepassing 

\section{Algemene verordening gegevensbescherming}
%Wanneer is gdpr van toepassing
De General Data Protection Regulation (GDPR) of in het Nederlands: Algemene Verordening Gegevensbescherming (AVG), is een nieuwe wetgeving die op 25 mei 2018 ingevoerd is met als doel regels op te stellen om de grondrechten en de fundamentele vrijheden van natuurlijke personen te beschermen in de Europese Unie, dit met name op hun recht op bescherming van persoonsgegevens. \autocite{avg2018privacy}

Voor de verwerking van deze gegevens worden maatregelen opgelegd over hoe deze op een correcte manier behandeld moeten worden en aan welke eisen een bedrijf moet voldoen indien deze hiermee handeld.

In dit onderdeel zal niet de volledige AVG beschreven worden, maar enkel de onderdelen m.b.t. een parkeersysteem met nummerplaatdetectie.

\paragraph{Persoonsgegevens}
In artikel 4 van het AVG worden persoonsgegevens als volgt beschreven: 'alle informatie over een geïdentificeerde of identificeerbare natuurlijke persoon ("de betrokkene"); als identificeerbaar wordt beschouwd een natuurlijke persoon die direct of indirect kan worden geïdentificeerd, met name aan de hand van een identificator zoals een naam, een identificatienummer, locatiegegevens, een online identificator of van een of meer elementen die kenmerkend zijn voor de fysieke, fysiologische, genetische, psychische, economische, culturele of sociale identiteit van die natuurlijke persoon'

Hieruit blijkt dat nummerplaten onder de term persoonsgegevens vallen; Deze zijn geregistreerd aan een persoon en kunnen worden gebruikt om de persoon te identificeren. In een parkeersysteem met nummerplaatdetectie zal hier dus ook op gelet moeten worden. 

Artikel 5 van GDPR, 'persoonsgegevens moeten worden verwerkt op een wijze die ten aanzien van de betrokkene rechtmatig, behoorlijk en transparant is ("rechtmatigheid, behoorlijkheid en transparantie").

\subsection{Voorwaarden van verwerking}
Vooraleer men verwerkingen mag uitvoeren op de persoonsgegevens van een gebruiker, moet deze gebruiker hier expliciete toestemming voor gegeven hebben.

\subsection{Rechten van de betrokkene}
De GDPR stelt enkele rechten op die de betrokkene, de eigenaar van de persoonsgegevens heeft. Deze zijn terug te vinden in hoofdstuk III van de GDPR.

\paragraph{Rechtmatigheid van verwerking}
De GDPR beschrijft in artikel 6 onder welke omstandigheden verwerking van persoonsgegevens rechtmatig is. Er zijn enkele omstandigheden beschreven, maar in de context van een ANPR-systeem kan men enkel persoonsgegevens verwerken indien de betrokkene toestemming gegeven heeft hiervoor. De verwerkingsverantwoordelijke moet later kunnen aantonen dat de betrokkene toestemming gegeven heeft gegeven.

\paragraph{Het recht om geïnformeerd te zijn}
Indien een ANPR-systeem is opgesteld, dient er volgens artikel 13 van de GDPR, gesignaleerd te worden dat een dergelijk systeem aanwezig is, dit tesamen met contactinformatie indien men meer informatie wenst te verkrijgen.

\paragraph{Recht op informatie en rectificatie van persoonsgegevens}
Iedereen bezit zijn eigen persoonlijke data en mag deze bijgevolg ook inkijken en corrigeren. Indien een gebruiker vraagt om zijn persoonsgegevens in te kijken, moet het bedrijf in kwestie al de persoonsgegevens van de gebruiker binnen de maand kunnen opleveren. Ook kan een gebruiker op eender wanneer beslissen om al zijn data te laten verwijderen. \autocite{avg2018privacy}

%\subsection{Data Protection Officer}
%Een Data Protection Officer (DPO) moet aangesteld worden, Nederlands: Functionaris voor gegevensbescherming. NIET NODIG VOOR KLEINE ORGANISATIE
\subsection{Verantwoordelijkheden van de verwerker}


\section{Belgische Camerawetgeving}
Sinds 25 mei 2018 is de nieuwe camerawetgeving ingevoerd, dit is een herziening van de Camerawetgeving uit 2007 en viel niet toevallig samen met de dag dat de GDPR is ingevoerd. Deze wet slaat op bewakingscamera's en geldt enkel wanneer deze als doel hebben:
\begin{itemize}
	\item Misdrijven tegen personen of goederen te voorkomen, vast te stellen of op te sporen;
	\item overlast in de zin van artikel 135 van de nieuwe gemeentewet, te voorkomen, vast te stellen of op te sporen, de naleving van gemeentelijke reglementen te controleren of de openbare orde te handhaven.
\end{itemize}
Aangezien nummerplaatdetectie als toegangssysteem geen van deze doelen bevat valt het niet onder de camerawetgeving voor bewakingscamera's. \autocite{staatsblad2007wet}

Natuurlijk zal er wel nog rekening gehouden moeten worden met de AVG omdat er persoonsgegevens worden verwerkt door een bedrijf, vereniging of eenmanszaak. \autocite{gba2019videoparlofoon}

%\subsection{Aangifte van camera's}
%Iedere bewakingscamera moet via het E-loket op http://www.aangiftecamera.be aangegeven worden. \autocite{besafe2018bewakingscameras}

%\subsection{Register}
%Sinds de GDPR die de privacywetgeving vervangt hoort er geen aangifte bij de Gegevensbeschermingsautoriteit plaats te vinden, maar wel de de verantwoordelijke voor de verwerking een een register bijhoudt  voor de verwerkingsactiviteiten die onder haar verantwoordelijkheid vallen. Dit register moet op verzoek in beschikking gesteld worden van de Gegevensbeschermingsautoriteit. \autocite{besafe2018beeldverwerking}

%In dit register wordt bijgehouden wat de doeleinden zijn %TODO

%\subsection{Openbare weg}
%Zolang de toegangscontrole op een niet publiek toegankelijke plaats is (Campus UGent). Dan is de camerawetgeving niet van kracht (?). Indien de camera een deel van de openbare weg waarneemt, dan moet er een aanvraag ingedient worden bij de politie, waarna dit wordt vastgelegd in de gemeenteraad. \autocite{beltug2018camerawet}

%Aangezien het een hele administratie is om dit in orde te brengen, kan het bedrijf het beeld van de openbare weg van de camera 'blacken'. Dit moet niet enkel op het beeld van de camera gebeuren, maar ook op de opgeslagen data. \autocite{beltug2018camerawet}

%\subsection{Bewaren van beelden}
%Bewaren van beelden mag maximaal 1 maand. 3 maanden in bedrijven met een verhoogd risico zoals luchthavens en havens.