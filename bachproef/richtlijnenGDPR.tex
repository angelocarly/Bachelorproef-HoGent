
\chapter{\IfLanguageName{dutch}{Wetgeving omtrent nummerplaatdetectie}{Technical details about ANPR in the GDPR}}
\label{ch:wetgeving-nummerplaatdetectie}

Het is vanzelfsprekend dat er enkele wetgevingen zijn die slaan op nummerplaatdetectie, het nemen van foto's als toegangssysteem is een grote stap op vlak van privacy in plaats van een token mee te geven. Zeker door nieuwe wet van de GDPR in 2018.

In volgende onderdelen wordt beschreven wat deze wetten inhouden en hoe deze op een nummerplaatdetectie slaan.

\section{Algemene verordening gegevensbescherming}
%Wanneer is gdpr van toepassing
De General Data Protection Regulation (GDPR) of in het Nederlands: Algemene Verordening Gegevensbescherming (AVG), is een nieuwe wetgeving die op 25 mei 2018 ingevoerd is met als doel regels op te stellen om de grondrechten en de fundamentele vrijheden van natuurlijke personen te beschermen in de Europese Unie, dit met name op hun recht op bescherming van persoonsgegevens. \autocite{avg2018privacy}

Voor de verwerking van deze gegevens worden maatregelen opgelegd over hoe deze op een correcte manier behandeld moeten worden en aan welke eisen een bedrijf moet voldoen indien deze hiermee handeld.

In dit onderdeel zal niet de volledige AVG beschreven worden, maar enkel de onderdelen m.b.t. een parkeersysteem met nummerplaatdetectie.

\subsection{Definities}

\paragraph{Persoonsgegevens}
In artikel 4 van het AVG worden persoonsgegevens als volgt beschreven: 'alle informatie over een geïdentificeerde of identificeerbare natuurlijke persoon ("de betrokkene"); als identificeerbaar wordt beschouwd een natuurlijke persoon die direct of indirect kan worden geïdentificeerd, met name aan de hand van een identificator zoals een naam, een identificatienummer, locatiegegevens, een online identificator of van een of meer elementen die kenmerkend zijn voor de fysieke, fysiologische, genetische, psychische, economische, culturele of sociale identiteit van die natuurlijke persoon' \autocite{avg2018privacy}

Hieruit blijkt dat nummerplaten onder de term persoonsgegevens vallen; Deze zijn geregistreerd aan een persoon en kunnen worden gebruikt om de persoon te identificeren. In een parkeersysteem met nummerplaatdetectie zal hier dus ook op gelet moeten worden. 

Artikel 5 van GDPR, 'persoonsgegevens moeten worden verwerkt op een wijze die ten aanzien van de betrokkene rechtmatig, behoorlijk en transparant is ("rechtmatigheid, behoorlijkheid en transparantie")` \autocite{avg2018privacy}.

\paragraph{Verwerking van gegevens}
Een verwerking is een veel vermelde term in het AVG, en omvat een duidelijke definitie. Het verwerken van gegevens omvat eender welke operatie die op persoonsgegevens wordt uitgevoerd. Waaronder het verzamelen, vastleggen, organiseren, raadplegen of vernietigen van deze gegevens onderdeel van is. Deze lijst wordt verder uitgebreid omschreven in Artikel 4 punt 2 van de GDPR.

Indien een ANPR-systeem foto's neemt van auto's dan voert deze enkele verwerkingen uit. Deze zijn het vastleggen, raadplegen en vernietigen van de gegevens. Ookal duurt dit proces maar enkele seconden is er wel degelijk sprake van verwerking.

Verder indien er een databank bijgehouden wordt van nummerplaten behoren er tot dit proces opnieuw een aantal bewerkingen. Deze zijn opnieuw het vastleggen, raadplegen en vernietigen.

\paragraph{Verwerkingsverantwoordelijke}
De verwerkingsverantwoordelijke is een persoon die instaat voor het correct verwerken van gegevens. Hij bepaalt het doel en de middelen van de verwerking. In andere woorden moet hij kunnen aantonen waarom en hoe de verwerking wordt uitgevoerd.

\paragraph{Verwerker}
Een verwerker is een persoon die persoonsgegevens verwerkt in opdracht van de verwerkingsverantwoordelijke. De verwerker bepaalt niet het hoe en waarom van de verwerking, maar is degene wie de verwerking uitvoert.

\subsection{Beginselen inzake gegevensverwerking}
Het is niet mogelijk om zomaar welke gegevens te verwerken. De GDPR stelt enkele regels op die organisaties moeten naleven voor het verwerken van gegevens. De verwerkingsverantwoordelijke moet kunnen aantonen dat deze nageleefd worden. De beginselen inzake gegevensverwerking zijn te vinden in Artikel 5 van de GDPR.

\paragraph{Rechtmatigheid van verwerking}
De GDPR beschrijft in artikel 6 onder welke omstandigheden verwerking van persoonsgegevens rechtmatig is. Indien de verwerking niet rechtmatig is kunnen er sancties optreden.

De GDPR beschrijft enkele voorwaarden waaraan minstens voldaan moeten worden. Zo kan de betrokkene toestemming gegeven hebben, is er een wettelijke plicht of zijn er gerechtvaardigde belangen voor de verwerking van de gegevens.

In het geval van een ANPR-camera is het niet mogelijk om toestemming aan de betrokkene te vragen. Dit omdat toevallige voorbijgangers ook de camera kunnen triggeren welke vervolgens hun gegevens verwerkt, waarvoor deze geen toestemming hebben kunnen geven. Hiervoor is de enige echte optie het gerechtvaardigd belang.

Het gerechtvaardigd belang berust erop dat de samenleving vindt dat het belang om de gegevens te verwerken zwaarwegend genoeg is dat ze de verwerking erkennen. En dat dit belang enkel kan behartigd worden door persoonsgegevens te verwerken \autocite{autoriteit2019gerechtvaardigd}.

\paragraph{Doelbinding}
De doeleinden voor de gegevensverwerking dienen uitdrukkelijk omschreven en gerechtvaardigd te zijn. Geen verdere verwerking is toegestaan die niet aan deze doeleinden beantwoordt.

\paragraph{Dataminimalisatie en juistheid}
De opgeslagen data dienen beperkt te zijn tot wat nodig is om de beschreven doeleinden te voldoen. Het is de bedoeling om een strikt minimum van data bij te houden. Verder moet de data actueel gehouden worden en indien onjuist, verwijdert of gerectificeerd te worden.

\paragraph{Opslagbeperking}
De data mag niet langer opgeslaan worden in een vorm waarbij de betrokkene identificeerbaar is dan noodzakelijk voor de verwezenlijking van de doeleinden.

\paragraph{Integriteit en vertrouwelijkheid}
De verwerking dient genoeg beveiligd te zijn door gebruik te maken van passende, technische organisatorische maatregelen. Verder dient deze beschermt te zijn tegen iedere niet toegelate of onwettige verwerking, tegen verlies, vernietiging of kwaliteitsverlies van de gegevens.

\paragraph{Verantwoordingsplicht}
De verantwoordelijke voor de verwerking/verwerker zijn verantwoordelijk voor de naleving van de beginselen van de verordening ongeacht het risiconiveau. Men moet kunnen aantonen dat de verwerking van de persoonsgegevens rechtvaardig verloopt.

\subsection{Rechten van de betrokkene}
\label{rechten-betrokkene}
Vervolgens stelt de GDPR enkele uitgebreide rechten op die de betrokkene, de eigenaar van de persoonsgegevens heeft. Deze zijn terug te vinden in hoofdstuk III van de GDPR en zijn omschreven in de volgende punten.

\paragraph{Recht om geïnformeerd te zijn}
Indien een ANPR-systeem is opgesteld, dient er volgens artikel 13 van de GDPR, gesignaleerd te worden dat een dergelijk systeem aanwezig is, dit tesamen met contactinformatie van de verwerkingsverantwoordelijke, de verwerkingsdoeleinden waarvoor de persoonsgegevens zijn bestemd en de rechtsgrond van de verwerking. Een voorbeeld van zo'n signalisatie is te zien in Figuur \ref{anpr-aanduiding}.

Deze signalisatie moet vanaf een duidelijke afstand zichtbaar zijn zodat de betrokkene deze kan lezen vooraleer hij de gemonitorde omgeving betreedt. Deze omgeving moet duidelijk zijn voor de betrokkene zodat hij deze kan mijden.

\begin{figure}[h!]
	\centering
	\includegraphics[width=\linewidth]{img/anpr-aanduiding.png}
	\caption{Voorbeeld van camerasignalisatie \autocite{edpb2019guidelines}}
	\label{anpr-aanduiding}
\end{figure}

\paragraph{Recht op informatie en rectificatie van persoonsgegevens} 
Iedereen bezit zijn eigen persoonlijke data en mag deze bijgevolg ook inkijken en corrigeren. Indien een gebruiker vraagt om zijn persoonsgegevens in te kijken, moet het bedrijf in kwestie al de persoonsgegevens van de gebruiker binnen de maand kunnen opleveren. Ook kan een gebruiker op eender wanneer beslissen om al zijn data te laten verwijderen \autocite{avg2018privacy}. Dit is terug te vinden in artikel 15 en 16 van de GDPR.

\paragraph{Recht op vergetelheid}
Een verwerkingsverantwoordelijke is verplicht persoonsgegevens zonder onredelijke vertraging te wissen indien deze niet meer nodig zijn voor de oorspronkelijke doeleinden, de gegevens onrechtmatig verwerkt zijn of de betrokkene zijn toestemming intrekt. Enkel indien één van de uitzonderingen in artikel 17 (3) van toepassing zijn kan dit langer duren. \autocite{edpb2019guidelines}

\paragraph{Recht op beperking van verwerking}
Bij camerasystemen die berusten op het gerechtvaardigd belang of openbaar belang om data te verwerken, heeft een betrokkene het recht om op eender welk moment hier bezwaar op te maken. Indien de verwerkingsverantwoordelijke geen legale gronden kan voorleggen die zwaarder doorwegen dan de rechten van de betrokkene, is hij verplicht aan de wensen van de betrokkene te voldoen. Dit is terug te vinden in artikel 18 van de GDPR.

In een context van camerabewaking kan deze objectie gemaakt worden voor, tijdens, of na een betrokkene een bewaakte zone betreedt. Dit betekent dat indien de belangen van de verwerkingsverantwoordelijke niet doorwegen tot de rechten van de betrokkene, de camera direct moet kunnen stopgezet worden van de betrokkene zijn data te kunnen verwerken. Anderzijds is het ook mogelijk om de gemonitorde zone genoeg af te bakenen zodat de verwerkingsverantwoordelijke de toestemming kan verifieren alvorens de betrokkene deze betreedt. \autocite{edpb2019guidelines}

%\subsection{Data Protection Officer}
%Een Data Protection Officer (DPO) moet aangesteld worden, Nederlands: Functionaris voor gegevensbescherming. NIET NODIG VOOR KLEINE ORGANISATIE

\subsection{Verwerking waarvoor identificatie niet is vereist}
Artikel 11 van de GDPR zegt dat indien de doeleinden niet vereisen dat de betrokkene geïdentificeerd is, dat de verwerker geen aanvullende gegevens hoeft bij te houden.
\par
Hieruit volgt dan ook dat indien de verwerker kan aantonen dat hij de betrokkene niet kan identificeren, artikelen 15 tot en met 20, beschreven in onderdeel \ref{rechten-betrokkene} niet meer van toepassing zijn.

\subsection{Functionaris voor gegevensbescherming}
Een functionaris voor gegevens bescherming of data protection officer (DPO) is iemand met deskundingheid op het gebied van gegevensbescherming. Een DPO wordt aangewezen om de verwerkingsverantwoordelijke of de verwerker te informeren en adviseren over hun verplichtingen over de GDPR en andere wetten over gegevensbescherming.

\paragraph{Voorwaarden voor het aanstellen}
Een DPO dient aangesteld te worden in volgende gevallen:
\begin{itemize}
	\item De verwerking wordt verricht door een overheidsinstantie of overheidsorgaan.
	\item De verwerker is hoofdzakelijk belast met verwerkingen die observatie op grote schaal van betrokkenen eisen.
	\item De verwerker is hoofdzakelijk belast met het verwerken van uitzonderlijke persoonsgegevens.
\end{itemize}

\paragraph{Taken}
De DPO dient de verwerkingsverantwoordelijke of de verwerker te informeren en adviseren over hun rechten en verplichtingen, het toezien op het naleven van de GDPR, samenwerken met de toezichthoudende autoriteit en optreden als contactpunt voor de toezichthoudende autoriteit.

\section{Toegepast op een ANPR-systeem}
Het ANPR-systeem dat Vado-Solutions voorstelt slaat geen persoonsdata op, de foto's die genomen worden om nummerplaten uit te lezen worden direct na de detectie verwijdert. Dit slaat niet af van het feit dat er verwerkingen op de foto's worden uitgevoerd en hier moet dus ook gepaste maatregelen op genomen worden.

Om deze foto's rechtmatig te kunnen verwerken is de enige echte optie 
ANPR gaat over het verzamelen van nummerplaatinformatie

enkel anpr gebruiken voor het juiste doel

indien de afbeeldingen niet opgeslaan worden, moet er verder geen zorgen gemaakt worden over de termijn hoelang data mag bijgehouden worden.

\section{Belgische Camerawetgeving}
Sinds 25 mei 2018 is de nieuwe camerawetgeving ingevoerd, dit is een herziening van de Camerawetgeving uit 2007 en viel niet toevallig samen met de dag dat de GDPR is ingevoerd. Deze wet slaat op bewakingscamera's en geldt enkel wanneer deze als doel hebben:
\begin{itemize}
	\item Misdrijven tegen personen of goederen te voorkomen, vast te stellen of op te sporen;
	\item overlast in de zin van artikel 135 van de nieuwe gemeentewet, te voorkomen, vast te stellen of op te sporen, de naleving van gemeentelijke reglementen te controleren of de openbare orde te handhaven.
\end{itemize}
Aangezien nummerplaatdetectie als toegangssysteem geen van deze doelen bevat valt het niet onder de camerawetgeving voor bewakingscamera's. \autocite{staatsblad2007wet}

Natuurlijk zal er wel nog rekening gehouden moeten worden met de AVG omdat er persoonsgegevens worden verwerkt door een bedrijf, vereniging of eenmanszaak. \autocite{gba2019videoparlofoon}

%\subsection{Aangifte van camera's}
%Iedere bewakingscamera moet via het E-loket op http://www.aangiftecamera.be aangegeven worden. \autocite{besafe2018bewakingscameras}

%\subsection{Register}
%Sinds de GDPR die de privacywetgeving vervangt hoort er geen aangifte bij de Gegevensbeschermingsautoriteit plaats te vinden, maar wel de de verantwoordelijke voor de verwerking een een register bijhoudt  voor de verwerkingsactiviteiten die onder haar verantwoordelijkheid vallen. Dit register moet op verzoek in beschikking gesteld worden van de Gegevensbeschermingsautoriteit. \autocite{besafe2018beeldverwerking}

%In dit register wordt bijgehouden wat de doeleinden zijn %TODO

%\subsection{Openbare weg}
%Zolang de toegangscontrole op een niet publiek toegankelijke plaats is (Campus UGent). Dan is de camerawetgeving niet van kracht (?). Indien de camera een deel van de openbare weg waarneemt, dan moet er een aanvraag ingedient worden bij de politie, waarna dit wordt vastgelegd in de gemeenteraad. \autocite{beltug2018camerawet}

%Aangezien het een hele administratie is om dit in orde te brengen, kan het bedrijf het beeld van de openbare weg van de camera 'blacken'. Dit moet niet enkel op het beeld van de camera gebeuren, maar ook op de opgeslagen data. \autocite{beltug2018camerawet}

%\subsection{Bewaren van beelden}
%Bewaren van beelden mag maximaal 1 maand. 3 maanden in bedrijven met een verhoogd risico zoals luchthavens en havens.