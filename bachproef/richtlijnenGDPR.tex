
\chapter{\IfLanguageName{dutch}{Wetgeving omtrent nummerplaatdetectie}{Technical details about ANPR in the GDPR}}
\label{ch:richtlijnengdpr}

Algemene Verordening Gegevensbescherming (AVG)

gemaakt om gegevens omtrent natuurlijke personen te beschermen en is van toepassing 

\section{Algemene verordening gegevensbescherming}
\subsection{Persoonlijke data}
Wat is persoonlijke data?
alle informatie over een geidentificeerde of identificeerbare natuurlijke persoon.
identificeerbaar is wanneer een persoon direct of indirect kan worden geidentificeerd aan de hand van een indicator zoals naam adres, geboortedatum.

Dit houdt dus ook in nummerplaat, aangezien deze gelinkt is aan een persoon.


Wat willen we opslaan van persoonlijke data?
Must: nummerplaat
naam?
gsm?

Artikel 5 van GDPR, 'persoonsgegevens moeten worden verwerkt op een wijze die ten aanzien van de brotkkene rechtmatig, behoorlijk en transparant is ("rechtmatigheid, behoorlijkheid en transparantie").

\paragraph{Inzage van gebruiksgegevens}
Iedereen bezit zijn eigen persoonlijke data en mag deze bijgevolg ook inkijken en corrigeren. Een bedrijf is verplicht de data van de desbetreffende gebruiker publiek te stellen. Ook kan een gebruiker op eender wanneer beslissen om al zijn data te laten verwijderen.

\section{Belgische Camerawetgeving}
Sinds 25 mei 2018 is de nieuwe camerawetgeving ingevoerd, dit is een herziening van de Camerawetgeving uit 2007 en viel niet toevallig samen met de dag dat de GDPR is ingevoerd. Deze wet slaat op bewakingscamera's en geldt enkel wanneer deze als doel hebben:
\begin{itemize}
	\item Misdrijven tegen personen of goederen te voorkomen, vast te stellen of op te sporen;
	\item overlast in de zin van artikel 135 van de nieuwe gemeentewet, te voorkomen, vast te stellen of op te sporen, de naleving van gemeentelijke reglementen te controleren of de openbare orde te handhaven.
\end{itemize}
Aangezien nummerplaatdetectie als toegangssysteem geen van deze doelen bevat valt het niet onder de camerawetgeving voor bewakingscamera's. \autocite{staatsblad2007wet}

Natuurlijk zal er wel nog rekening gehouden moeten worden met de AVG omdat er persoonsgegevens worden verwerkt door een bedrijf, vereniging of eenmanszaak. \autocite{gba2019videoparlofoon}


%\subsection{Aangifte van camera's}
%Iedere bewakingscamera moet via het E-loket op http://www.aangiftecamera.be aangegeven worden. \autocite{besafe2018bewakingscameras}

%\subsection{Register}
%Sinds de GDPR die de privacywetgeving vervangt hoort er geen aangifte bij de Gegevensbeschermingsautoriteit plaats te vinden, maar wel de de verantwoordelijke voor de verwerking een een register bijhoudt  voor de verwerkingsactiviteiten die onder haar verantwoordelijkheid vallen. Dit register moet op verzoek in beschikking gesteld worden van de Gegevensbeschermingsautoriteit. \autocite{besafe2018beeldverwerking}

%In dit register wordt bijgehouden wat de doeleinden zijn %TODO

%\subsection{Openbare weg}
%Zolang de toegangscontrole op een niet publiek toegankelijke plaats is (Campus UGent). Dan is de camerawetgeving niet van kracht (?). Indien de camera een deel van de openbare weg waarneemt, dan moet er een aanvraag ingedient worden bij de politie, waarna dit wordt vastgelegd in de gemeenteraad. \autocite{beltug2018camerawet}

%Aangezien het een hele administratie is om dit in orde te brengen, kan het bedrijf het beeld van de openbare weg van de camera 'blacken'. Dit moet niet enkel op het beeld van de camera gebeuren, maar ook op de opgeslagen data. \autocite{beltug2018camerawet}

%\subsection{Bewaren van beelden}
%Bewaren van beelden mag maximaal 1 maand. 3 maanden in bedrijven met een verhoogd risico zoals luchthavens en havens.