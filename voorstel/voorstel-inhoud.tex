%---------- Inleiding ---------------------------------------------------------

\section{Introductie} % The \section*{} command stops section numbering
\label{sec:introductie}

Parkings zijn een belangrijk onderdeel van het dagelijkse leven. Iedere dag rijden talloze wagens door hun toegangspunten om na een drukke werkdag weer verlaten te worden. Ieder van deze wagens zal zich dan ook moeten identificeren om toegang te kunnen verkrijgen tot hun parking. Dit is een reden van opstroppingen bij drukke toegangspunten, maar ook vervuiling van de omstreken door het weggooien van tickets.

Hedendaags zijn er al vele alternatieve oplossingen voor de hand om toegang te regelen. Zo gebruikt men de standaard methoden zoals tickets, maar ook NFC chips op badges en zelfs smartphones en nummerplaatdetectie worden gebruikt. Deze technologieen brengen vraag naar welke van deze de beste optie voordraagt bij het gebruik van parkeergarages.

Dit onderzoek wordt uitgevoerd met oog op de parking van UGent, waar men nood heeft aan een nieuw toegangssysteem. Momenteel worden er tokens gebruikt om de parking te verlaten, maar dit brengt enkele negatieve punten mee. Zo worden de tokens snel kwijtgeraakt en zijn deze duur om te maken. Enkele oplossingen werden bekeken om dit slimmer aan te pakken, maar een grote favoriet is het gebruik van alpr camera's die via een centraal systeem bepaalde wagens toegang kunnen verlenen. Hierbij moet wel rekening gehouden worden met wetgevingen zoals de GDPR. Verder zal bekeken worden op welke hardware deze camera's kunnen werken. Zo zal er vanuit een financieel perspectief gekozen worden voor Raspberry Pi 3 B+ met Pi-cam waarop open-source software van openALPR zal gerunt worden. Dit moet vlot werken en indien niet, zal er geopteerd moeten worden voor andere hardware.

Zo komen we aan volgende onderzoeksvragen:
\begin{itemize}
  \item Welke toegangstechniek brengen het meest profijt voor de parkings van UGent?
  \item Is nummerplaatdetectie een haalbare techniek omtrent privacy en GDPR?
  \item Kan men openalpr runnen op een Raspberry Pi 3 B+ met Pi-cam?
\end{itemize}

%---------- Stand van zaken ---------------------------------------------------

\section{State-of-the-art}
\label{sec:state-of-the-art}


Hedendaags kampt UGent met verscheidene problemen omtrent hun huidig toegangssysteem. Tokens worden gebruikt om de parkeergarage te kunnen verlaten, maar deze hebben een grote kans om verloren te geraken. Wat zorgt voor een grote kost om deze bij te laten maken, en een negatieve impact op het milieu. Verder zijn deze tokens universeel en kunnen in eender welke tokenslikker ingevoerd worden.
Verder zijn er enkele andere toegangsmethoden voorgesteld waaronder, het gebruik van briefjes met een geprinte barcode. Dit zou het probleem van de hoge kost van de tokens verhelpen, maar is milieubelastend en men moet nog steeds de slikkers legen.
Nog een mogelijke techniek zou het gebruik van ALPR-camera's zijn die m.b.v. een centraal systeem toegang kan verlenen aan bepaalde auto's. Dit zou geen milieubelasting veroorzaken maar heeft wel een hoge opzetkost. Ook moet er onderzocht worden hoe de privacy-wetgeving hierop invloed speelt en hoe dit aangepakt kan worden.

Dit onderzoek zal nagaan welke toegangstechnieken het meest voordelig zijn om uit te werken in deze case aan UGent. Dit door een vergelijkende studie op vlak van: benodigde werkuren, miliebelastbaarheid, transparantie voor opvolging en een betere controle op de toegang. Verder zal er uitgebreid gekeken worden hoe ALPR-camera's gebruikt kunnen worden zodat deze niet ten strijde gaan met wetgevingen zoals de privacy-wetgeving en de GDPR. Allerlaats wordt ook gekeken of deze software voor ALPR uitgevoerd kan worden op een kleine microcontroller zoals de Raspberry pi 3 B+.

% Voor literatuurverwijzingen zijn er twee belangrijke commando's:
% \autocite{KEY} => (Auteur, jaartal) Gebruik dit als de naam van de auteur
%   geen onderdeel is van de zin.
% \textcite{KEY} => Auteur (jaartal)  Gebruik dit als de auteursnaam wel een
%   functie heeft in de zin (bv. ``Uit onderzoek door Doll & Hill (1954) bleek
%   ...'')

%---------- Methodologie ------------------------------------------------------
\section{Methodologie}
\label{sec:methodologie}

Vooraleer de onderzoeksvragen beantwoordt worden is er nood aan inzicht in verschillende mogelijke toegangstechnieken voor parkings. Dit zal gedaan worden a.d.h.v. een literatuurstudie, waarbij dan ook de eerste onderzoeksvraag zal beantwoordt worden. In deze literatuurstudie zullen de karakteristieken op vlak van milieuvriendelijkheid, gebruiksvriendelijkheid en kost vergeleken worden. Vervolgens zal hieruit de keuze gemaakt worden welke techniek gewenst is voor een grote parkeergarage met meerdere toegangspunten.

Om de tweede onderzoeksvraag te kunnen beantwoorden zal nog een literatuurstudie uitgevoerd worden omtrent privacy en GDPR. Met als doel richtlijnen te bekomen voor het legaal gebruik van camera's op een parking.

Voor de laatste onderzoeksvraag zal onderzocht worden of OpenALPR een mogelijke optie is om te gebruiken op een Raspberry Pi 3B+. Er zullen foto's genomen worden van voertuigen aan de toegangspunten van UGent, waarna er getest wordt of deze nummerplaten detecteerbaar zijn met de Raspberry Pi.

%---------- Verwachte resultaten ----------------------------------------------
\section{Verwachte resultaten}
\label{sec:verwachte_resultaten}
Er wordt verwacht dat in de case van UGent, nummerplaatdetectie het meest profijtelijk is. Dit doordat er een groot aantal toegangspunten zijn tot de parking waar enkel alpr-camera's en controllers moeten komen. Ook is de netwerkinfrastructuur al gelegd rond de campus. Het implementeren van andere technieken zoals tickets zouden wel een verbetering leveren, maar hebben nog steeds onderhoud nodig zoals de slikkers legen.

%---------- Verwachte conclusies ----------------------------------------------
\section{Verwachte conclusies}
\label{sec:verwachte_conclusies}


Conclusie: alpr